\documentclass{report}
\usepackage{float}
\usepackage[utf8]{inputenc}
\usepackage[T1]{fontenc}
\usepackage[ngerman]{babel}
\usepackage{graphicx,wrapfig,lipsum}
\usepackage{geometry}
\usepackage{minted}
\usepackage{xcolor}
\usepackage{csquotes}
\usepackage{dirtree}
\usepackage{hyperref}
\usepackage{fancyhdr}
\usepackage{makecell}
\usepackage{array}
\usepackage[toc,page]{appendix}
\usepackage[backend=biber,style=ieee]{biblatex}  % or style=numeric, authoryear, etc.
\addbibresource{main.bib}
\geometry{top=2.5cm, bottom=2.5cm, left=2.5cm, right=2.5cm}



% -------------------------------------------------------
% Titelseiten-Informationen (Diese Abändern)
% -------------------------------------------------------
\title{Erstellung einer Todo App mit React und AWS CDK}

% Art der Arbeit + Studiengang
\newcommand{\arbeitstyp}{Beleg}
% Modul / Lehrveranstaltung + Dozent:innen
\newcommand{\modul}{Modul / Lehrveranstaltung: WebWeb Engineering 3\\
Dozent: Christopher Hilgner}
% Hochschule + Fakultät
\newcommand{\hochschule}{
\includegraphics[height=2.5cm]{LOGO_HSZG_SUBLINE_GRANIT.png}\\[5mm]
Hochschule Zittau/Görlitz \\ Fakultät Elektrotechnik und Informatik
}

% Autor
\author{
    Niklas Kaulfers \\[2mm]
    Matrikelnummer: 1064032
}
% Betreuer (für BA-Thesis, sonst leer lassen)
\newcommand{\betreuer}{}

% Datum der Abgabe
\date{Abgabedatum: \today}

% -------------------------------------------------------
% Titelseite neu definieren
% -------------------------------------------------------
\makeatletter
\renewcommand{\maketitle}{
    \begin{titlepage}
        \centering
        {\Large \hochschule\par}
        \vfill

        {\huge \bfseries \@title\par}
        \vspace{1cm}

        {\large \arbeitstyp\par}
        \vspace{0.5cm}

        {\large \modul\par}
        \vspace{0.5cm}

        {\large \betreuer\par}
        \vspace{1.5cm}

        {\Large \@author\par}
        \vspace{2cm}

        {\large \@date\par}

        \vfill
    \end{titlepage}
}
\makeatother


\fancypagestyle{plain}{%
    \fancyhead[C]{%
        \includegraphics[height=1.25cm]{LOGO_HSZG_SUBLINE_GRANIT_ICON_ONLY.png}%
    }
    \fancyfoot[L]{\sffamily Kaulfers}
    \fancyfoot[C]{\sffamily \arbeitstyp}
    \fancyfoot[R]{\sffamily\thepage}
    \renewcommand{\footrulewidth}{0.4pt}
}

\begin{document}

\pagestyle{plain}
\pagenumbering{gobble}
\setlength{\headheight}{47pt}

\maketitle
\tableofcontents
\pagenumbering{arabic}


\chapter{Einleitung}
    Ein durchaus häufig erstelltes Projekt in der Softwareentwicklung ist eine Todo-App.
    Anhand solcher können grundlegende Kenntnisse der Entwickler zur jeweiligen Umgebung gezeigt werden.
    Im Verlauf dieser Arbeit wird der Entwicklungsprozess einer solchen App im Serverless Kontext mit einem React Frontend erläutert.

\chapter{Tech Stack}
\section{Backend}
    Das Backend besteht vollständig aus AWS Servicen.
    Um genau zu sein ApiGateway, Lambda, DynamoDB und Cognito.
    Diese haben jeweils eigene Aufgaben.
    So ist ApiGateway verantwortlich dafür, eine RestAPI zur Verfügung zu stellen.
    Alle Endpunkte der Anwendung sind innerhalb dieser API zu finden.
    Der Service Cognito schützt die Endpunkte, in dem er nur verifizierten Nutzern den Zugriff auf die API gestattet.
    Ohne Verifizierung hat der Nutzer des Frontends keinen Zugriff auf die API Endpunkte.
    Lambdas stellen Funktionen im Serverless Kontext da.
    In diesen kann die Logik der Anwendung mit eigenem Code definiert werden.
    Innerhalb der hier behandelten Anwendung wird die gesamte Logik in Typescript und mittels Nodejs geschrieben.
    Für persistenz sorgt DynamoDB, die NoSQL Datenbank von AWS.\@
\section{Frontend}
    Um mit dem Backend zu interagieren wird ein React Frontend verwendet.
    Mit React können Single Page Applications (SPAs) erstellt werden.
    Eine solche SPA besteht aus mehreren Komponenten, welche individuell, voneinander unabhängig ausgeführt werden können.
    Dieses nutzt für styling \textit{Tailwindcss} und \textit{MUI}.\@
    Tailwind stellt eine Sammlung an css Code da, welcher im Code einfach durch HTML Klassen genutzt werden kann.
    MUI hingegen ist eine Komponenten Bibliothek für React. 
    Zusammen sorgen MUI und Tailwind für ein anschauliches Erscheinungsbild.\\
    Für Nutzerverifzierung im Frontend wird \textit{react oidc} genutzt.
    

\chapter{Fazit}

\begin{appendices}
\end{appendices}
\printbibliography\end{document}